
\section{Teinilaulu}

Pojat tenttiin emme me ehtineet\\
kera veljien uurastaneitten.\\
Meill' eivät laakerit lehtineet\\
akateemisten seppeleitten.\\
Mutt' laulu on meidän ja laulupuu,\\
se kukkii kun laakerit lakastuu.

Me saavuimme kerran Helsinkiin\\
hyvin nuorina, voittoisina.\\
Pian palattais taas, niin päätettiin,\\
joka ainoa maisterina.\\
Muut tulivat silloin ja tekivät niin.\\
Me jäimme jäljelle Helsinkiin.

Toki meillekin riemu ja rikkaus työn\\
oli tuttua, koimme sen kyllä.\\
Mutt' valkeus yhden keväisen yön,\\
yks' \sout{neitonen} syleilyllä,\\
tuhat kertaa suuremman riemun toi.\\
Luvut jäivät ja kirjoja nakersi koi.

Niin vuodet vierivät luotamme pois\\
kuin virtana viivähtämättään.\\
Moni meistä jo tahtonut mukaan ois,\\
mutt' nousta ja nostaa kättään,\\
ei jaksanut heikko ja herkkä mies,\\
joka elämän riemut jo tarkkaan ties.

Mutt' laulumme, teinilaulumme tää\\
oli meidän ja rakkaus neitten.\\
Ne taisivat täysin ymmärtää\\
teot teinien eksyneitten.\\
Ne shamppanjaks elon kalkin loi\\
ja laulumme, teinilaulumme soi.

Ja kun kerran, veljet, kaikki on nää\\
vain unohtunutta multaa,\\
ylioppilaslakkimme jälkeen jää,\\
– hiven lyyrassa kiiltää kultaa.\\
Näin teinit, toverit eli ja joi,\\
ja elämän kirjaa nakersi koi.